\documentclass{beamer}
%\documentclass[handout]{beamer}
%\usepackage{pgfpages}
%\pgfpagesuselayout{4 on 1}[a4paper,border shrink=5mm,landscape]
\usepackage[italian]{babel}
\usepackage[utf8]{inputenc}
\usepackage[T1]{fontenc}
\usepackage{lmodern}
\usepackage{verbatim}
\usepackage{amsmath}
\usepackage{amsfonts}
\usepackage{amssymb}
\usepackage{graphicx}
\usepackage{listings}
\usepackage{hyperref}
\usepackage{multirow}

\definecolor{mygray}{rgb}{0.5,0.5,0.5}
\definecolor{RaspberryPi}{HTML}{BB1142}

\usetheme{Pittsburgh}
\setbeamercovered{dynamic}
\setbeamertemplate{frametitle}[default][left]

\setbeamercolor{structure}{fg=RaspberryPi}
%\setbeamercolor{frametitle}{fg=OpenStack}
%\setbeamercolor{title in head/foot}{fg=OpenStack}
%\setbeamercolor{author in head/foot}{fg=blue!55!black}
%\setbeamercolor{date}{fg=blue!40!black}
%\setbeamercolor{institute in head/foot}{fg=blue!55!black}
%\setbeamercolor{section in head/foot}{fg=blue!55!black}

\setbeamertemplate{sidebar canvas right}{\vspace*{3pt}\hspace*{-93pt}{\includegraphics[angle=0,origin=c,height=40pt]{imgs/logo02.png}}}
\beamertemplatenavigationsymbolsempty


\begin{document}


\begin{frame}[plain]
\begin{center}
\includegraphics[width=0.08\textwidth]{imgs/logo-uniroma2-red.png}
\\    
\textbf{\color{RaspberryPi} Università  degli studi di \\Roma Tor Vergata}
\\[0.2cm]
\tiny Facoltà di Ingegneria
\\[0.2cm]
\small Roma2LUG
\\
\tiny Linux User Group
\\[0.2cm]


\vfill
{
\Large \textbf{Roma2LUG Incontra}
\\[0.2cm]
\color{RaspberryPi}\Large \textbf{Music On Linux}
\\[1.0cm]
}
\vfill

\small{
\textbf{Speaker}
\hfill
\textbf{Speaker}
\\
\textit{Giulia Cassarà}
\hfill
\textit{Emanuele Savo}}


\end{center}

\end{frame}


%\begin{frame}
%\frametitle{\textbf{Indice della presentazione}}
%\framesubtitle{Argomenti trattati}
%\tableofcontents
%\end{frame}

\lstdefinestyle{customc}{
  belowcaptionskip=1\baselineskip,
  breaklines=true,
  %language=BASH,
  showstringspaces=false,
  frame=none,
  basicstyle=\small\ttfamily,
  keywordstyle=\bfseries\color{blue},
  commentstyle=\itshape\color{red},
  stringstyle=\color{orange},
  rulecolor=\color{black}
}

\lstset{escapechar=@,style=customc}


%Inizio presentazione

\begin{frame}
\frametitle{\textbf{OpenStack}}
\framesubtitle{\textbf{Esempio di topologia virtuale realizzabile sulla piattaforma IaaS}}
\begin{figure}
\centering
\includegraphics[scale=0.15]{imgs/rasp3b.jpg}
\begin{block}{Studio Piattaforma IaaS}
\begin{itemize}
\item[$\bullet$] Born as a MiniPC
\item[$\bullet$] Can reproduce HD movies
\item[$\bullet$] The main difference with a PC are the GPIO ports
\end{itemize}
\end{block}
\end{figure}
\end{frame}

%%%%%%%%%%%%%%%%%%%%%

\begin{frame}
\frametitle{\textbf{OpenStack}}
\framesubtitle{\textbf{Esempio di topologia virtuale realizzabile sulla piattaforma IaaS}}
\begin{figure}
\centering
\includegraphics[scale=0.70]{imgs/rasp3btec.png}
\end{figure}
\end{frame}

%%%%%%%%%%%%%%%%%%%%%%

\begin{frame}
\frametitle{\textbf{OpenStack}}
\framesubtitle{\textbf{Esempio di topologia virtuale realizzabile sulla piattaforma IaaS}}
\begin{figure}
\centering
\includegraphics[scale=1]{imgs/gpio.png}
\end{figure}
\end{frame}

%%%%%%%%%%%%%%%%%%%%%%%%%%%%%%

\begin{frame}[fragile]
\frametitle{\textbf{Suite di benchmark CRD (Cpu Ram Disk)}}
\framesubtitle{\textbf{Loader \& Worker}}

\begin{block}{AAAAAAAAAAAAAAAAAAAAAAAAAA}
\begin{itemize}
\item[$\bullet$] Download Raspbian OS for the Raspberry Pi
\begin{lstlisting}
  $ wget https://downloads.raspberrypi.org/raspbian_lite_latest
\end{lstlisting}
\item[$\bullet$] Unzip Raspbian OS for the Raspberry Pi
\begin{lstlisting}
  $ unzip xxxx-xx-xx-raspbian-jessie-lite.zip
\end{lstlisting}
\end{itemize}
\end{block}

\end{frame}

%%%%%%%%%%%%%%%%%%%%%%%%%%%%%%%

\begin{frame}[fragile]
\frametitle{\textbf{Suite di benchmark CRD (Cpu Ram Disk)}}
\framesubtitle{\textbf{Loader \& Worker}}
\begin{block}{Studio Piattaforma IaaS}
\begin{itemize}
\item[$\bullet$] Insert SD card
\item[$\bullet$] Search for device name of the SD card with this command:
\begin{lstlisting}
  $ sudo fdisk -l
\end{lstlisting}
\item[$\bullet$] Search for info about your SD card. \textit{Warning, be careful!}
\begin{lstlisting}[basicstyle=\tiny\ttfamily]
  Disk /dev/mmcblk0: 14,5 GiB, 15523119104 bytes, 30318592 sectors      
  Units: sectors of 1 * 512 = 512 bytes                                 
  Sector size (logical/physical): 512 bytes / 512 bytes                 
  I/O size (minimum/optimal): 512 bytes / 512 bytesa                    
  Disklabel type: dos                                                   
  Disk identifier: 0x6f92008e                                           
\end{lstlisting}
\item[$\bullet$] Replace mmcblk0 with device name of your SD
\begin{lstlisting}
  $ sudo dd \
    if=/xxxx-xx-xx-raspbian-jessie-lite.img \
    of=/dev/mmcblk0
\end{lstlisting}
\end{itemize}
\end{block}

\end{frame}

%%%%%%%%%%%%%%%%%%%%%%%%%%%%%%

\begin{frame}[fragile]
	\frametitle{\textbf{Suite di benchmark CRD (Cpu Ram Disk)}}
	\framesubtitle{\textbf{Loader \& Worker}}

	\begin{block}{Studio Piattaforma IaaS}
		\begin{itemize}
			\item[$\bullet$] Connect ethernet cable to the Raspberry Pi
			\item[$\bullet$] Connect HDMI cable to the Raspberry Pi
			\item[$\bullet$] Connect micro USB power cable to the Raspberry Pi
			\item[$\bullet$] Waiting for coplete boot...
			\item[$\bullet$] Login
			\begin{itemize}
				\item[$\bullet$] user: pi
				\item[$\bullet$] password: raspberry
			\end{itemize}
			\item[$\bullet$] Execute these commands:
			\begin{lstlisting}
  $ sudo apt-get update
  $ sudo apt-get dist-upgrade -y
  $ sudo apt-get install rpi-upate -y
			\end{lstlisting}
		\end{itemize}
	\end{block}
\end{frame}

%%%%%%%%%%%%%%%%%%%%%%%%%%%%%%%

\begin{frame}[fragile]
	\frametitle{\textbf{Suite di benchmark CRD (Cpu Ram Disk)}}
	\framesubtitle{\textbf{Loader \& Worker}}

	\begin{block}{Studio Piattaforma IaaS}
		\begin{itemize}
			\item[$\bullet$] Config Raspbian OS with this tool
			\begin{lstlisting}
   $ sudo raspi-config
			\end{lstlisting}
			\begin{itemize}
				\item[$\bullet$] Expand Filesystem
				\item[$\bullet$] Internationalisation Options
				\begin{itemize}
					\item[$\bullet$] Change Locale
					\item[$\bullet$] Change Timezone
					\item[$\bullet$] Change Keyboard Layout
					\item[$\bullet$] Change wifi Country
				\end{itemize}
			\end{itemize}
			\begin{lstlisting}
   $ sudo reboot
			\end{lstlisting}
			\item[$\bullet$] Update Raspberry Pi firmware
			\begin{lstlisting}
   $ sudo rpi-update
   $ sudo reboot
			\end{lstlisting}
		\end{itemize}
	\end{block}
\end{frame}

%%%%%%%%%%%%%%%%%%%%%%%%%%%%%%

\begin{frame}[fragile]
	\frametitle{\textbf{Suite di benchmark CRD (Cpu Ram Disk)}}
	\framesubtitle{\textbf{Loader \& Worker}}

	\begin{block}{Studio Piattaforma IaaS}
		\begin{itemize}
			\item[$\bullet$] Install library for gpio and other tools
			\begin{lstlisting}
  $ sudo apt-get install wiringpi git vim
			\end{lstlisting}
			\item[$\bullet$] Download the scripts
			\begin{lstlisting}
  $ git clone https://github.com/Roma2Lug-Projects/MusicOnLinux.git
			\end{lstlisting}
			\item[$\bullet$] Open the script
			\begin{lstlisting}
  $ cd MusicOnLinux
  $ vim keyboard.sh
  $ vim smario.sh
			\end{lstlisting}
		\end{itemize}
	\end{block}
\end{frame}

%%%%%%%%%%%%%%%%%%%%%%%%%%%%%%%

\begin{frame}[fragile]
	\frametitle{\textbf{Suite di benchmark CRD (Cpu Ram Disk)}}
	\framesubtitle{\textbf{Loader \& Worker}}

	\begin{block}{Studio Piattaforma IaaS}
		\begin{itemize}
			\item[$\bullet$] Give execute permission
			\begin{lstlisting}
  $ chmod +x keyboard.sh
  $ chmod +x smario.sh
			\end{lstlisting}
			\item[$\bullet$] Execute the scripts!
			\begin{lstlisting}
  $ ./keyboard.sh
  $ ./smario.sh
			\end{lstlisting}
		\end{itemize}
	\end{block}
\end{frame}


%%%%%%%%%%%%%%%%%%%%%%%%%%%%%%%

\begin{frame}[fragile]
	\frametitle{\textbf{Keyboard.sh}}
			\begin{lstlisting}[language=bash]
  #! /bin/bash
  tone () {
  local note="$1"
  local duration="$2"
  if test "$note" -eq 0; then
    gpio -g mode 18 in
  else
    local period="$(perl -e"printf'%.0f',600000/440/2**(( $note-69)/12 )")"
    echo $period
    gpio -g mode 18 pwm
    gpio pwmr "$(( period ))"
    gpio -g pwm 18 "$(( period/2 ))"
    gpio pwm-ms
    sleep $duration
    tone 0
  fi
}

			\end{lstlisting}
\end{frame}


%%%%%%%%%%%%%%%%%%%%%%%%%%%%%%%

\begin{frame}[fragile]
	\frametitle{\textbf{Keyboard.sh}}
		\begin{lstlisting}[language=bash]
  #! /bin/bash
  local note="$1"
  local duration="$2"
  if test "$note" -eq 0; then
    gpio -g mode 18 in
  		\end{lstlisting}
\end{frame}

%%%%%%%%%%%%%%%%%%%%%%%%%%%%%%%


\begin{frame}[plain]
\begin{center}
\includegraphics[width=0.12\textwidth]{imgs/logo-uniroma2-red.png}
\vfill
\huge{\textit{\textbf{Grazie per l'attenzione}}}
\vfill
\includegraphics[width=0.40\textwidth]{imgs/logo02.png}
\end{center}
\end{frame}

\end{document}